\section{Introduction}
%This document represents a template of the assignment of the course \textit{Immigration to software systems and services} at the GSSI\cite{gssi}.

%The total length of this document must not exceed 20 pages, including references, appendixes, \etc

%In this section you have to provide (i) the informal description of the system you are going to model (ii) the main motivation behind the AADL analysis you chose, (iii) brief description of the obtained results, and (iii) brief description of your Acceleo program.  

In the United States, over 3,000 people per year lose their lives to fires \cite{usfiredeaths}. Technology can be a powerful tool to reduce this number and increase the safety of buildings and their inhabitants. In this assignment, we propose a model of a smart fire alarm system, as well as analysis of certain aspects of the model and an example of tool that could be developed with the model.

In order to detect fires and alert the appropriate authorities, this system is formed of sensors, communication devices, infrastructure, an alarm and a database. In broad terms, environmental data is obtained by the sensors, which is sent to the mobile phones of local users and to a remote control center, which can analyze the data and dispatch emergency services. The model for this system is developed using the Architecture Analysis and Design Language (AADL) \cite{aadl}, which allows us to model hardware and software aspects of a multi-device system.

Considering the time sensitive nature of fire detection, prevention and combat, we decided that the latency of the system, that is, the time it takes for data to be sent from one component to another, is the most important aspect to be analyzed. Using OSATE \cite{osate}, a tool developed for AADL development, latency analysis was performed on the model, measuring time between production and consumption of data from different devices. The analysis reports were used to display and understand the latency of the system.

Finally, the model is used as basis for an Acceleo \cite{acceleo} project which takes an XMI file produced by the AADL model as input and produces a web page displaying components from the system.

The remainder of this document is organized as follows: chapter 2 explains the AADL model produced; chapter 3 proposes an alternative structure to the model and shows the analysis results; chapter 4 elaborates on the Acceleo template used to generate a web page using the AADL model as input; finally, chapter 5 provides final thoughts on the assignment.

%\limit{1}